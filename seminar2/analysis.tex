\section{Analysis}
To help in the analysis a few papers have been used as references. First the
article {\em "What Makes Good Research in Software Engineering"} by Mary Shaw
\cite{shaw2002}. The second paper is the {\em "Case Studies for Method and Tool
Evaluation"} by Barbara Kitchenham, Lesley Pickard and Shari Lawrence Pfleeger
\cite{kitchenham1995}. The third and last is the {\em "Empirical Research in
Software Engineering"} by Claes Wohlin, Martin H\"{o}st and Kennet Henningsson
\cite{wohlin2003}.

\subsection{The Question}
The question is never clearly written down in a sentence or two, though from the
context it's easy to understand that the purpose with this paper is to prove that
the Extreme Programming method is a viable choice as a development method. This
fits under the question category, "Method or means of development" from Shaws
report \cite{shaw2002}. Abrahamsson set out to answer this question by
gathering as much quantative data as possible which hadn't been done before.

\subsection{The Result}
The results presented in the paper can be categorised as "Answer/Judgment or a
report" in the types of results that Shaw presents \cite{shaw2002}. This
categorisation is mainly done by exclusion of the other "types of results". As
a logical answer to the question asked would be the actual procedures that
someone would take to introduce Extreme programming methadology for a development team but that is not what is
given in this paper instead it relies on the knowledge that other papers and books
present that information. Another category that comes close is the "Empirical
model" though no model is presented only empirical data.

\subsection{The Validation}
With a question and results some validation must be done. Abrahamsson goes
through his research method thoroughly and towards the end gives a
comprehensive discussion as well as listing all the conclusions without generalising
too heavily. I categorise this as persuasion when it comes to Shaws types of
validation \cite{shaw2002}. Abrahamsson also states that the findings here are
interesting and serves its purpose but {\em "[...] will continue to follow up
    the development process and will report the concrete results from the
project as a whole [...]"}. This indicates that the conclusions from the
    case study are interesting but should not be generalised. The short time
    span from which the data is collected makes it hard to generalise.

\section{Some weaknesses}
The case study doesn't generalise too much and it presents some interesting
findings though it has some weaknesses as well. This is especially true if you
compare the case study with the case study guidelines presented by Kitchenham
et. al. \cite{kitchenham1995}. The most obvious part is that Abrahamsson
doesn't compare his results with any previous work or a "company baseline".
Perhaps the exact same results would have been collected if they e.g. used
Scrum as a development method or even some iterative version of the V-model
development method.

As follow up to the lack of baseline, what kind of baseline project Abrahamsson should
have used is hard to say. The reason for this is that it's presented in the
paper that they have already bought the software from an external company which
indicates that they don't have much development in-house. So without
development in-house a baseline project would perhaps consist of a team of
students with similiar experience that would use some other kind of development
method, developing the same or similiar software. The problem that would arise here is
to limit the confounding factors, even if students with similiar experience
would be chosen for such a project, all developers have different experience
that could influence the results in the end.

Some issues that are not answered in the introductory parts of the report are
the questions about how much the development team knew beforehand. Did they
know that they were a part of a research study? If they knew, were they
positive or skeptical towards it? According to Kitchenham et. al. {\em "Staff
    morale can have a large effect on productivity and quality. [...] To
    minimise this effect, you must staff a case-study project using your normal
staff-allocation method."} As Abrahamsson hired students specifically for this
project it seems like he actually followed the normal staff-allocation method
for the research institute. All this comes down to how far the case study will
be generalised inside the research institue or even outside.

Another issue is that even if the comparison between release one and release two is deemed valid as a
research comparison perhaps the next release would have shown completely
different results. Abrahamsson mentions that the experience from previous
research say that it's the first release cycle that is the biggest learning step
and then it settles. Is it for sure that release two isn't a learning step as
well for these students? What if it shows that the improvement from release two
to three is as big as the one from release one to two. That would basically
invalidate the entire result from this paper.

One way Abrahamsson could have planned this case study a little better is to
have done the case study with the software development company that were
already developing the main part of the application (not just a subset of
features). If the research institute could have hired the developers from that
company instead of students to develop the same features in a subsystem that
they had already done, a comparison could have been made between their
development methadology and the XP methadology used the second time around.
This is what Kitchenham et. al. \cite{kitchenham1995} calls "Replicated Product
Design". The result from a case study like that would probably have been more
interesting than the case study done.
\newpage
