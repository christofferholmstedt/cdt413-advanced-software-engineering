\newpage
\section{Summary of the paper}
The paper "Extreme Programming: First Results from a Controlled Case Study"
\cite{abrahamsson2003} tries to deal with the lack of papers concerning Extreme
programming (XP) practices that include quantitative results. Abrahamsson says that
{\em Successful XP adoptions have however been criticized for the lack of
concrete data.} Abrahamsson set up a controlled study at a large Finnish
research institute by hiring four developers for a fixed period of eight weeks.
The task for the developers were to use XP practices while developing a small
program for managing large amount of research data.

The eight week fixed release schedule was divided into five different releases.
This case study goes through the result from the first two releases. The first
one released after week 2 and the second one released after week 4. To tackle
the problem with lack of quantiative data in researh papers simliar to the one
from Abrahamsson, he set out to analyse the difference between release one and
two when it comes to "estimation accuracy" in percent, productivity in lines of
code/hour (LOC/h) and post-release defect rate as defects per thousand lines of
code (defects/KLOC).

The developers were co-located with the customer in this case Abrahamsson
himself which is typical for XP. The developers were all students with a few
years experience from the industry though all of them were given two basic
books as well as a short introductory course in XP to give them a head start
before the actual project began. It's also made clear from Abrahamsson that
the case study only deals with a development team that is new to XP.

The most important results presented in the paper are the collected data from
release 1 and 2 as aimed for. For release 1 team productivity was 13.39
loc/h and defects/KLOC was 2.19. For release 2 team productivity was 25.12 and
defects/KLOC was 2.10. The estimation accuracy went from -48\% to
-22\%. Another interesting finding for the agile movement was the customer
involvement didn't have too big as big as previously thought. Over the course
of all four weeks only 5.3\% of the customer's time was spent with the project.
