\newpage
\section{Scrum}
As with XP, Scrum has several parts which relates back to the first principle.
The most obvious ones in Scrum are the initial planning phase (creation of
product backlog), sprint planning meetings and the daily meetings. In the
initial planning phase all stakeholders including customer and developers get
together to create the product backlog. The customer comes up with stories
(requirements in Scrum are called stories) and the development team estimate
how big each story is. Next up is the sprint planning phase. The customer now
selects stories in priority while the development team warns the customer if
too
many stories are selected for the upcoming sprint. The last part is the daily
scrum meetings where the scrum master and developers meet to discuss what
has been completed since last meeting, discuss any obstacles if any as well as
what is expected to be completed today. All these parts in Scrum are
face-to-face meetings to convey information.

Rising and Janoff \cite{rising00} doesn't mention the Scrum retrospective
\cite{web:scrumalliance-101} which is the regular
meeting after each sprint that deals with process and methadology questions
such as how to make development even more effecient. This is the most obvious
part of Scrum that relates back to the second principle.
