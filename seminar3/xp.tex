\newpage
\section{XP}
Beck \cite{beck99} lists early on the major practices in XP. The two parts that relates back to the
first principle is "Pair programming" and "On-site customer". In non agile
development processes the customer is rarely on-site and rarely visits the
software development company only when necessary or when it's time for
delivery of a product. This is in contrast to the XP practice to always have
the customer close by. A practice such as this one makes it obvious that you
prefer face-to-face conversations over any other type of conversation.

The pair programming concept within a team also highlights the face-to-face
conversation. At first it can seem that you will only do half of the work if
two developers works together but the constant face-to-face communication can make
the development process so effective that it's worth it.

Beck also mentions that pair programming is a way for new employees to get to
know the codebase and learn how to work within the company. This ones again put
focus on that it's worth to devote a developer full time to teach a new
employee by constantly working together and having face-to-face conversations with each
other.

When it comes to the second principle it's a little bit harder to identify
parts within XP. The list of XP practices written by Beck \cite{beck99} doesn't
have any obvious practice that tackles the problem to regularly devote time to
improve the development process.

One part barely related to the second principle is the concept of "40-hours weeks". If
one or more developers work too much there is a problem behind it that needs to
be adressed. If a development team stay faithful to this princpiple, meetings
will occur when someone works too much to adress the problems, though not
on a reguarly interval. At least this could be viewed as that there is some kind of safety net to
detect some problems in XP.
