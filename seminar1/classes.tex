\newpage
\section{Classes}
\subsection{What is implemented}
In this chapter a thorough explanation of all classes and their role in the
kwicPrototype application will be given. An UML class diagram can be viewed
in the end of this report that shows the architecture on a class level. Chosen
alternative to look at is alternative A, decomposition structure and the
{\em is-a-submodule-of} relation \cite{book:bass2012}.

The first trivial requirement is that no matter how input or output of information
is handled there must be some kind of way to work with the data (lines of text)
while adding features to the software e.g. some kind of storage of the data
in memory in a List, HashMap or perhaps Array. To hide exactly how this is done
in the kwicPrototype an interface for this was created, the "LineStorageInterface".
In the kwicPrototype their is an implementation of this as "ListLineStorage" and
the possiblity in the future to add other implementations if needed.

After deciding how the lines were to be stored while doing different operations
on it the next step were to decide on how to import the lines. Given as requirement
for laboration 1 was that import and export to txt files was a must-have feature but
the possiblity for future extensions to store in other formats was also required.
Instead of separating import and export into different parts it was decided a
common Input/Output interface would be a good choice. A future extension with
support for a new file format would probably implement both import and export
at the same time so it makes sense to keep that funtionality together. This
common functionality was organised as "IOInterface" and txt implementation in
"TxtFileIO".

With both hiding of "data storage" and "input/output" functionality taken care of
it was time to look at the algorithm to do circular shift and
sorting the lines.

A problem can usually be solved with several different algorithms.
To be able to choose which algorithm to use, some requirements needs to be put
on the software. Should performance/speed be of highest importance or perhaps
disk space is a limitation?
The chosen algorithm one day may be the wrong one next week when new requirements
from a stakeholder are presented, it's apparent that this must be able to change
in the future. The "CircularlyShiftedAlgorithmInterface" takes care of this and
hides that functionality behind its method "compute". In kwicPrototype a prototype
of this algorithm was implemented in "CSAPrototype".

The last requirement in laboration 1 was to sort all lines. Sorting is a known and common
problem for all software developers and sorting can be done in many ways. The
actual sorting of lines is hidden in kwicPrototype behind the interface "SortLinesInterface".

To summaries this section about kwicPrototype implementation the LineStorageInterface
hides the functionality on how the lines are stored while running the application e.g. Lists or Arrays?
The IOInterface hides the functionality on how import and export is handled, new file
formats can be added as example. The CircuarlyShiftedAlgorithmInterface hides the
shifting implementation and SortLinesInterface hides the sorting functionality.
All these interfaces have at least one corresponding implementation which is a
submodule of its interface.

\subsection{What is not implemented}
The current implementation imports the information from txt file and modifies
that object as the program progresses. This can be a problem in the future
if their is a need for keeping track of changes made or someone just want to keep
the original object while performing the CSA algorithm as well as sorting on a
clone.

A solution to this could be to implement an abstract LineStorage that implements
functionality such as "clone" and "equals". This would make it possible to clone
imported rows as well as comparing different LineStorage object with each other.
This functionality was not implemented in Laboration 1 as it was decided this would not
be needed.
