\begin{abstract}
Through time computer networks have evolved into fault tolerant systems, that have substantial level of reliability.
Still they are far from perfect, and there can be numerous potential threats that can affect their proper functioning. 

%Constructing high quality threat models and studying their effect on different types of computer networks can be a very complex process that requires a substantial amount of time and financial funds.
%The threat models used in this paper have been composed by the authors to simulate the possible circumstances that are likely to occur.
%By applying them on centralised and decentralised computer networks, it is possible to analyze their effect on the specific network architectures or applications.

The main focus of this paper are centralised and decentralised network architectures. 
In order to test their resistance to threats, three threat models have been composed.
%Special emphasis has been put on how centralised and decentralised computer network deal with infrastructure and security threats mainly related with the users data.
The threat models are constructed in that fashion to target the infrastructure and user's data in the networks.
First threat model presented in the paper is the \textit{Natural disaster} which affects the infrastructure. 
The other two threat models are \textit{Privacy} and \textit{Censorship} which can be very harmful to the user's data.

By applying these threat models potential flaws into the architecture can be revealed.
Results obtained from these newly occured situations will be used to summarise which network architecture is more resistant to a specific threat. 
As a conclusion from our work, we can say that decentralised systems have shown better results dealing with these threats.
Decentralised systems are more tolerant to infrastructure failures.
Also by keeping the user's data in distributed environment censorship and privacy threats can be avoided.
%As a conclusion, a summary of how the threat models changed the characteristics of the centralised and decentralised networks and propose best solution in situations with high probability of occurrence of a particular threat model.
\end{abstract}
